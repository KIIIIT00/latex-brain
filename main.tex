\documentclass{classes/report}

\begin{document}

\pagenumbering{roman}
\subfile{sections/title}
\newpage

% \tableofcontents
\clearpage
% \pagenumbering{arabic}
\textbf{\large 1.選んだトピックとその背景}

私が選んだ1つ目のトピックは,生きた動物の脳深部における特定の神経細胞の活動を精密に制御し,観察できる点や光が神経細胞を活性化し,その反応をイメージングするという視覚的な表現に変換する点に興味を持った.
また,電気的性質を用いた能動的な観測方法の制限はどのようなものかについて疑問を持った.


\textbf{\large 2.AIツールによる初期的な整理}

使用したAIツールは,Geminiであり,使用した日付は,2025年7月11日である.1つ目のトピックについて生成結果をまとめたものを以下に示す.
2光子顕微鏡は,イメージングだけでなく,特定の神経細胞群を選択的に刺激する光遺伝学と組み合わせることで,神経回路の刺激と応答の同時観察を可能にする.
また,この顕微鏡は,近赤外線を利用して,生きた動物の最大1mmの深部まで細胞構造や機能のイメージングが可能である.

\textbf{\large 3.文献・論文による補強・修正}

チャネルロドプシンの作用スペクトルと蛍光体の励起スペクトルは重なることがあるが,2光子顕微鏡を用いることで,蛍光体の励起に必要な光の強度を最小限に抑え,スキャン顕微鏡法を使用することで,イメージング中のチャネルロドプシンの不要な光活性化を抑えることができる\cite{YIZHAR20119}.
チャネルロドプシンは,微生物に由来する光受容タンパク質の一種で,光に応答してイオンチャネルを開き,細胞の膜電位を変化させることで神経活動を制御している\cite{Xu02012020}.

\textbf{\large 4.自分の考察・理解の変化}

私は,生きた動物の脳深部における神経細胞活動を精密に制御・観察できるという技術的な側面に興味を持っていた.しかし,文献調査を通じて,この技術が2光子顕微鏡と光遺伝学という技術の組み合わせによって実現されていることを理解した.
特に,チャネルロドプシンによって光に応答してイオンチャネルを開き,細胞の膜電位を変化させることで神経活動を制御するという具体的なメカニズムを知ることができた.
また,私がいていた「電気的性質を用いた能動的な観測方法の制限」という疑問に対しても、明確な答えを得ることができた。蛍光体の励起スペクトルとチャネルロドプシンの作用スペクトルが重なるという技術的課題が存在することを学んだ。しかし同時に、2光子顕微鏡を用いることで蛍光体励起に必要な光強度を最小限に抑え、スキャン顕微鏡法によってイメージング中のチャネルロドプシンの不要な光活性化を抑制できることも理解した。これにより、当初考えていた「制限」は、適切な技術の組み合わせによって克服可能であることが分かった。
最も重要な理解の変化は、この技術が単なる観察手法ではなく、神経回路の刺激と応答を同時に観察できる双方向的な研究手法であるという点である。これは、神経科学研究において因果関係を直接的に検証できる強力なツールであることを意味している。

\newpage

\textbf{\large 1.選んだトピックとその背景}

私が選んだ2つ目のトピックは,最適ビン幅は具体的にどのような基準で最適と判断されるか,またビン幅が不適切だとどのような問題が生じるかについて疑問を持った.

\textbf{\large 2.AIツールによる初期的な整理}

使用したAIツールは,Geminiであり,使用した日付は,2025年7月11日である.2つ目のトピックについて生成結果をまとめたものを以下に示す.
スパイクヒストグラムを作成するときに,時間を区切るためのビン幅を設定し,時間分解能と発火率分解能の2つの要素の間でトレードオフの関係にある.ビン幅が大きすぎる場合は,ヒストグラムは過度に平滑化され,発火率の時間的な変化を捉えることができない.
一方で,ビン幅が小さすぎる場合は,ヒストグラムはノイズによる大きな変動を示し,本来の発火率のパターンが不明瞭になる.

\textbf{\large 3.文献・論文による補強・修正}

スパイクヒストグラムにおけるビン幅を,推定されるレートと未知の基底レートとの間の平均二状誤差を最小化する原理に基づいて選択する方法を提案している\cite{10.1162/neco.2007.19.6.1503}.
一方で,同じ最適化原理をカーネル密度推定に適用し,カーネルのバンド幅を選択する手法\cite{Shimazaki2010}を提案している.スパイクヒストグラムのビン幅とカーネル密度推定のバンド幅は,どちらもスパイクから発火率を推定する際の平滑化パラメータである.


\textbf{\large 4.自分の考察・理解の変化}

最適ビン幅の具体的な基準については、スパイクヒストグラムのビン幅を推定されるレートと未知の基底レートとの間の平均二状誤差を最小化する原理に基づいて選択する方法が存在することを学んだ.
これにより、最適性の判断が主観的なものではなく、客観的な数学的基準に基づいていることを理解できた。
また、AIツールの説明で得た「時間分解能と発火率分解能のトレードオフ」という概念により、ビン幅選択の本質的なジレンマをより深く理解することができた。ビン幅が大きすぎると過度の平滑化により時間的な変化を捉えられず、小さすぎるとノイズによる変動で本来のパターンが不明瞭になるという具体的な問題を把握できた。
さらに、スパイクヒストグラムにおけるビン幅の問題が、カーネル密度推定におけるバンド幅選択という、より一般的な平滑化パラメータの選択問題として理解できることも学んだ。同じ最適化原理がカーネル密度推定にも適用されているという事実は、神経科学における解析手法が継続的に発展し、より洗練された方法へと進化していることを示している。
これらの学習を通じて、両トピックが単なる技術的な問題ではなく、神経科学研究の根幹に関わる重要な課題であることを認識した。特に、適切な観測・解析手法の選択が、神経活動の理解において決定的に重要であることを深く理解することができた。

\textbf{\large 5.参考文献リスト}

\printbibliography[heading=none]

\end{document}